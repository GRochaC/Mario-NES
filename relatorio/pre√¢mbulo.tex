% Thiago T. P, 2022/2
% template original: https://www.overleaf.com/read/yfjhddwvpcgh
%%%%%%%%%%%%%%%%%%%%%%%%%%%%%%%%%%%%%%%%%%%%%%%%%%%%%%%%%%%%%%%%%
%% LEMBRE-SE DE INCLUIR OS ARQUIVOS preâmbulo.tex E risc-v.tex %%
%%%%%%%%%%%%%%%%%%%%%%%%%%%%%%%%%%%%%%%%%%%%%%%%%%%%%%%%%%%%%%%%%
\usepackage[T1]{fontenc} % hifenização (quebra de palavra)
\usepackage[utf8]{inputenc} % caracteres acentuados
\usepackage[brazil]{babel} % tradução inglês -> português; e. g., resumo ao invés de abstract

\usepackage{graphicx}	% possibilita inserção de imagem/pdf no texto
\usepackage{rotating}	% permite a inserção de figuras deitadas
\usepackage{float}		% dá a opção [H] para inserção precisa de figura
\usepackage{booktabs}	% fornece comandos úteis para a confecção de tabelas
\usepackage{multirow}	% para mesclagem de células verticais numa tabela

\usepackage{amsmath, amssymb}	% suporte para equações matemáticas 

\usepackage{mathptmx}		% fonte times no corpo de texto
\usepackage[scaled]{helvet}	% helvetica em seções, subseções, legendas, etc.

\usepackage{titlesec} % pacote para modificações nas strings de seções e similares
\titleformat*{\section}			{\bfseries\sffamily\scshape}	% altera a fonte e estilo do nome de seção
\titleformat*{\subsection}		{\bfseries\sffamily\scshape}	% altera a fonte e estilo do nome de subseção
\titleformat*{\subsubsection}	{\sffamily\scshape}				% altera a fonte do nome de subsubseção
% 0 pt -> identação esquerda, *2 -> espaço acima do nome, *.5 -> espaço abaixo
\titlespacing*{\section}		{0pt}{*2}{*.5}	% modifica o espaçamento do nome de seção
\titlespacing*{\subsection}		{0pt}{*2}{*.5}	% modifica o espaçamento do nome de subseção
\titlespacing*{\subsubsection}	{0pt}{*2}{*.5}	% modifica o espaçamento do nome de subsubseção

% definição de vários comprimentos importantes da página e texto
%% comprimentos em polegadas (in) para ter numeros redondos
\setlength\paperheight 		{11in}		% altura da folha
\setlength\paperwidth  		{8.5in} 	% largura da folha
\setlength{\textheight}		{9.25in}	% altura do texto
\setlength{\topmargin}		{-0.7in}    % reajuste de margens
\setlength{\headheight}		{0.20in}
\setlength{\headsep}		{0.25in}
\setlength{\footskip}		{0.50in}
\flushbottom
\setlength{\textwidth}		{7in}		% largura do texto
\setlength{\oddsidemargin}	{-0.25in}
\setlength{\evensidemargin}	{-0.25in}
\setlength{\columnsep}		{2pc}		% separação entre colunas (1 in ~ 6 pc)

\usepackage{caption} 	% permite mudar configurações de legendas
\captionsetup{
	labelfont={sf, bf},	% legenda em helvetica e negrito
	textfont=it,		% texto da legenda em itálico
	format=hang			% linhas identadas (observável apenas em textos longos)
}

\usepackage{lastpage}	% cria macro para o número da última página
\usepackage{fancyhdr}	% fornece comandos que facilitam enfeitar a página
	%% informações no topo 
	\fancyhead[L]{Turma 3} 							% à esquerda
	\fancyhead[C]{2022/2}							% ao centro
	\fancyhead[R]{Prof. Dr. Marcus Vinicius Lamar}	% à direita
	%% informações no fundo
	\fancyfoot[L]{CIC0099 -- Organização e Arquitetura de Computadores} % à esquerda
	\fancyfoot[C]{}														% ao centro (nenhuma)
	\fancyfoot[R]{\thepage/\pageref*{LastPage}}							% à direita
	                       %^ número da última página 
	%% linhas horizontais superior e inferior
	             % comprimento   % altura
	\renewcommand{\headrulewidth}{0.5pt}	% mude para 0 pt se quiser apagar a linha
	\renewcommand{\footrulewidth}{0.5pt}	% mude para 0 pt se quiser apagar a linha 
	
	\pagestyle{fancy} % doravante, todas a páginas, exceto o título, terão os enfeites definidos acima

\usepackage{xcolor}		% maior variedade de cores
\usepackage{listings} 	% inserção de texto de código em Latex
	\input{risc-v}		% suporte para o Assembly RISC-V

% comando \teaser para a criação da figura opcional no título
\newcommand{\teaser}[2]{ % recebe dois argumentos, uma figura e uma legenda
	\begin{center}
		\captionsetup{type=figure}
		\includegraphics[width=5cm]{#1} % figura é o primeiro argumento,
		\captionof{figure}{#2}			% legenda é o segundo
	\end{center}
}

\usepackage[colorlinks]{hyperref} % este pacote gera a funcionalidade de hyperlinks, citações, etc. e para evitar erros deve ser carregado por último